%\documentclass[11pt,a4paper,landscape,twocolumn]{article}
\documentclass[11pt,a4paper]{article}

\usepackage[utf8]{inputenc}
\usepackage{amsmath}
\usepackage{amsfonts}
\usepackage{amssymb}
\usepackage{graphicx}
\usepackage{xcolor}
\usepackage{listings}
\usepackage{hyperref}

\usepackage{palatino}


% ligne de séparation entre les colonnes
\setlength{\columnseprule}{0.4pt}

% espace inter-colonnes
\setlength{\columnsep}{20pt}

\lstset{
	language=C++
	,showlines=true
	,showstringspaces=true
	,tabsize=2
	,frame=single
	,breaklines=true
	,framexleftmargin=2mm
	,backgroundcolor=\color{white}
	,xleftmargin=2mm
	%, frameround={tttt}
	,captionpos=b
	,basicstyle=\tt \scriptsize
	,keywordstyle=\bfseries\ttfamily\color[rgb]{0,0,1}
	,numberstyle=\scriptsize,
	,identifierstyle=\ttfamily
	,commentstyle=\color[rgb]{0.133,0.545,0.133}
	,stringstyle=\ttfamily\color[rgb]{0.627,0.126,0.941}
	,showstringspaces=false
	,numbers=none
%	numbers=left,
	,stepnumber=1
	,numbersep=10pt
	,tabsize=4
	,breaklines=true
	,breakatwhitespace=false
	columns=flexible
%	upquote=true,
	extendedchars=true
	,xleftmargin=0pt
	,language=bash
   ,frame=single
}

%\usepackage{draftwatermark}
%\SetWatermarkColor{gray!20}
%\SetWatermarkFontSize{1cm}
%\SetWatermarkText{Version préliminaire !}
%\SetWatermarkScale{2}

\usepackage[printwatermark]{xwatermark}
%\newwatermark*[allpages,color=red!10,angle=45,scale=2,xpos=0,ypos=0]{Version préliminaire !}

\usepackage[left=2.00cm, right=1.50cm]{geometry}
\author{S. Kramm}
\title{C++ Techniques avancées}
\date{\today}
\begin{document}
\maketitle

\begin{abstract}
Ce document présente quelques techniques avancées de C++.
Il utilise principalement les avancées permises par C++11.
\end{abstract}

\section{Rappels: templates}

On peut définir un gabarit (on utilisera ci-après le terme de {\em template}) de fonction ou de classe.

\begin{lstlisting}
template<typename T>
void foo( T t );
\end{lstlisting}

On pourra alors utiliser cette fonction pour différent types:
\begin{lstlisting}
foo( 42 );
foo( 3.14 );
foo( Machin() ); // appel d'un constructeur
\end{lstlisting}





\section{Introduction}

\subsection{Rappel sur le choix de la fonction par le compilateur}

Différence overload - instanciation de template

Lors de l'appel d'une fonction, le compilateur est capable de déterminer automatiquement la bonne fonction en fonction du type des arguments:
\begin{lstlisting}
void foo( int a );  // fonction 1
void foo( float a); // fonction 2
...
foo( 4 ); // appel de la fonction 1
foo( 1.23 ); // appel de la fonction 2
\end{lstlisting}

Sauf si la différence est sur le type de retour:
\begin{lstlisting}
int   foo();  // fonction 1
float foo(); // fonction 2
...
int a = foo(); // impossible d'avoir une resolution automatique!
\end{lstlisting}

Dans le cas de template de fonctions, ce mécanisme de sélection intervient {\bf après} l'instanciation des templates.


\section{En pratique: associé a enable\_if}

Principe:
le trait \fbox{\tt std::enable\_if<condition, T>::type}%
\footnote{Par la suite, et afin d'alléger l'écriture, on omettra le préfixe {\tt std::}}
sera équivalent à \fbox{\tt T} si "condition" est "true".
Sinon, cela donnera une construction invalide et la fonction sera rejetée du jeu de d'overload, {\bf sans génération d'erreur}.

Par exemple, ceci:
\begin{lstlisting}
std::enable_if<condition, int>::type foo();
\end{lstlisting}
Si "condition" est vrai (valeur booléenne {\em true}), alors ce sera équivalent à ceci:
\begin{lstlisting}
int foo();
\end{lstlisting}

Cette condition peut s'utiliser
\begin{itemize}
	\item soit dans la liste des paramètres de type:
\begin{lstlisting}
template<typename T1,typename T2,enable_if<condition,int>* = nullptr>
foo();
\end{lstlisting}

	\item soit à la place d'un type, dans la valeur de retour ou dans la liste des arguments:
\begin{lstlisting}
template<typename T1>
foo(T1 a, enable_if<condition,int> b);
\end{lstlisting}
	
\end{itemize}

\begin{lstlisting}
template< bool B, class T = void >
struct enable_if;
\end{lstlisting}

Principe:
{\em if B is true, {\tt std::enable\_if} has a public member typedef type, equal to T; otherwise, there is no member typedef. }

Si B est "true" (obtenu par exemple par {\tt std::is\_same<T,int>::value}, qui sera "true" pour T=int), alors le type fournit un typedef {\tt type}, qui sera égale à {\tt T}.


\section{Exemples complets}


\subsection{Exemple 1}
Fonction qui renvoie un type dépendant du type en template

\begin{lstlisting}
template<typename T>
typename std::enable_if<
	( std::is_same<T, int>::value || std::is_same<T, char>::value ), T
	>::type
myFunc(T data)
{
	std::cout << "myFunc() data=" << data << "\n";
	return data+1;
}

int main()
{
    std::cout << myFunc( 42 );  // ok, affiche 43
    std::cout << myFunc( 'a' );  // ok, affiche 'b'
    std::cout << myFunc( 3.14 );  // Erreur, ne compile pas !      
}
\end{lstlisting}



\subsection{Exemple 2}
Fonction avec un argument

\begin{lstlisting}
struct A
{
    int a;
    A() { a=42; }
};

// for A& or const A&
template<
	typename T,
	typename std::enable_if<
		(std::is_same<T,A>::value || std::is_same<T,const A>::value),T
	>::type* = nullptr
> 
A&
func ( A& a1, T& a2 )
{
    std::cout << "template 1\n";
    return a1;
}

// for all other types

template<
	typename T,
	typename std::enable_if<
		(!std::is_same<T,A>::value & !std::is_same<T,const A>::value),T
	>::type* = nullptr
> 
A&
func ( A& a1, T a2 )
{
    std::cout << "template 2 a2=" << a2 << "\n";
    return a1;
}
 
int main()
{
    A a1,a2;
    A a3 = func(a1,a2);
    A a4 = func(a1,3);
}
\end{lstlisting}

\subsection{Exemple 3}

Soit deux types sur lesquels on veut définir un opérateur de multiplication, qui renverra le second:
\begin{lstlisting}
struct A
{
 ...
};
struct B
{
...
};
int main()
{
	A a; B b:
	B b2 = a * b;
}
\end{lstlisting}

\begin{lstlisting}
\end{lstlisting}


Mais en réalité, on en manipule plusieurs B, qui sont stockés dans un vecteur, et on veut pouvoir les multiplier tous avec un A, sans avoir à écrire nous même la boucle.
Concrètement, on veut faire:
\begin{lstlisting}
int main()
{
std::vector<B> vb;
auto vb2 = A * b;
}
\end{lstlisting}

Il faut donc définit un opérateur

\section{Cas pratiques}

\subsection{Stream operators and constness}

Say you have some type
\begin{lstlisting}
struct A
{
	A( int v) : val(v) {}
	int val=42;
};
\end{lstlisting}

And you want a stream operator, so you can print it out.
There you go:

\begin{lstlisting}
A& operator( std::ostream& f, const A& a)
{
	f << a.val;
};
\end{lstlisting}

Of course, you need to declare it as friend in the class:
\begin{lstlisting}
struct A
{
	friend A& operator( std::ostream&, const A& );
	A( int v) : val(v) {}
	int val=42;
};
\end{lstlisting}

But lets say that when streaming it, you also need to edit its value, say increment it by one.
With the above implementation, you can't: the argument is {\tt const}.

So you need another implementation:
\begin{lstlisting}
A& operator( std::ostream& f, A& a )
{
	f << a.val++;
};
\end{lstlisting}

The class becomes:
\begin{lstlisting}
struct A
{
	A( int v) : val(v) {}
	friend A& operator( std::ostream&, const A& );
	friend A& operator( std::ostream&, A& );
	int val=42;
};
\end{lstlisting}

But then, one may wonder if the first one really needed ?
The answer is yes, because if we remove it, the code below would be illegal:
\begin{lstlisting}
	std::cout << A(123);
\end{lstlisting}


\subsection{Factorization}
\label{ssec:facto1}

Supposons qu'on aie deux surcharges d'une fonction avec la même signature, la seule différence étant que l'argument de l'une d'entre elle est "const".
Le code des deux fonctions peut-être différent.
%Say I have two overloads on the same function, one for const arguments, the other for non const.
\begin{lstlisting}
void foo( A& a )
{
	// ...
	a.doSomething();
}
void foo( const A& a )
{
	// ...
}
\end{lstlisting}

Ces deux fonction peuvent être appelée directement depuis un code utilisateur, ou depuis une autre fonction "bar()", avec la même signature.
Cette dernière fonction devrait - a priori - donc être dupliquée également:
%These two functions can be called from user code, or from a "bar library function, thats therefore needs to be duplicated:
\begin{lstlisting}
void bar( A& a )
{
//    code 1
// ......
	foo( a )
}

void bar( const A& a )
{
//    code 1
// ......
	foo( a )
}
\end{lstlisting}

Si maintenant on se rend compte que la partie dans "code 1" est identique, on voit que l'on fait de la "réplication de code", ce qui est en général mauvais signe.
Comment factoriser et éviter cette réplication?
Il suffit simplement de transformer les deux fonction "bar()" en un template de fonction, qui sera instancié en deux versions, une pour un argument "const", et l'autre pour un argument "non-const", et qui appelera la fonction "foo()":

%Now the part in "code 1" is actually the same, and code duplicating is bad.
%So you can factorize that in some helper function, so I can directy write:

\begin{lstlisting}
template<typename T>
void bar( T& a )
{
    std::cout << "bar(), is_const=" << std::is_const<T>::value << "\n";
//    code 1    
	foo(a);
}
\end{lstlisting}

Le code complet est visible en \ref{lst:lst03}.

\lstinputlisting[float,caption={voir section \ref{ssec:facto1}},label=lst:lst03]{code/sfinae_03.cpp}


This will print:
\begin{lstlisting}
helper, is_const=0
foo( A& a )
helper, is_const=1
foo( const A& a )
\end{lstlisting}

Version avec 2 arguments:

\begin{lstlisting}
void foo( A& a1, A& a2 )
{
    std::cout << "foo( A& a )\n";
    a2.val++;
}
void foo( A& a1, const A& a2 )
{
    std::cout << "foo( const A& a )\n";
}

template<typename T1, typename T2>
void helper_func( T1& a1, T2& a2  )
{
    std::cout << "helper, T2 is_const=" << std::is_const<T2>::value << "\n";
	foo( a1, a2 );
}

void bar( A& a1, A& a2 )
{
	helper_func( a1, a2 );
}

void bar( A& a1, const A& a2 )
{
	helper_func( a1, a2 );
}
 
int main()
{
    A a1, a_noconst;
    const A a_const;
    bar( a1, a_noconst );
    bar( a1, a_const );
}
\end{lstlisting}

With returned argument:


\section{Tag dispatching}

La technique du {\em tag dispatching}
permet d'implémenter un comportement différent en fonction du type de données à manipuler.
Le principe est d'offrir dans l'API une fonction d'interface, qui ne va que "dispatcher" à des sous-fonctions.
La sélection se fera de façon automatique lors de la compilation en fonction du type de l'argument.
Cette sélection est basée sur le principe de surcharge de fonction:
les sous-fonctions auront dans leur signature un argument supplémentaire "muet" permettant au compilateur de faire la sélection.
Par exemple:

\begin{lstlisting}
template<typename T>
void foo( T t )
{
	impl_foo( t, /* ici un type deduit de T */  )
}

template<typename T>
void impl_foo( T t, int )
{
	// code pour int
}

template<typename T>
void impl_foo( T t, float )
{
	// code pour float
}
\end{lstlisting}


Pour bien clarifier l'intention, on introduit une couche de séparation de ces fonctions.
S'il s'agit de fonctions libre, on les place dans un espace de nom distinct:
\begin{lstlisting}
template<typename T>
void foo( T t )
{
	detail::impl_foo( t, /* ici un type deduit de T */  )
}

namespace detail {
	template<typename T>
	void impl_foo( T t, int )
	{
		// code pour int
	}
	
	template<typename T>
	void impl_foo( T t, float )
	{
		// code pour float
	}
}
\end{lstlisting}

S'il s'agit de fonctions membre de classe, alors on les mets dans la partie "private":

\begin{lstlisting}
struct MyClass
{
	public:
		template<typename T>
		void foo( T t )
		{
			impl_foo( t, /* ici un type deduit de T */  )
		}
	private:
		template<typename T>
		void impl_foo( T t, int )
		{
			// code pour int
		}
		
		template<typename T>
		void impl_foo( T t, float )
		{
			// code pour float
		}
};
\end{lstlisting}


L'aspect "tag" vient de ce que l'on prévoit des types vides qui ne serviront que d'identifiant:
\begin{lstlisting}
struct Type_A {};
struct Type_B {};
\end{lstlisting}

Le dispatch se fera via une classe "helper" (on omet ici ):
\begin{lstlisting}
template<typename T>
struct Helper {};

template<typename T>
void foo( T t )
{
	detail::impl_foo( t, Helper<T> )
}

template<typename T>
void impl_foo( T t, Helper<Type_A> )
{
	// code pour le type A
}
template<typename T>
void impl_foo( T t, Helper<Type_B> )
{
	// code pour le type T = B
}
\end{lstlisting}




A DEVELOPPER

ref:
\begin{itemize}
\item \url{https://en.wikibooks.org/wiki/More_C%2B%2B_Idioms/Tag_Dispatching}
\item \url{https://www.fluentcpp.com/2018/04/27/tag-dispatching/}
\end{itemize}

\section{AAAA}

\section{References}


references:
\begin{itemize}
\item \url{https://en.cppreference.com/w/cpp/types/enable_if}
\item \url{https://en.wikipedia.org/wiki/Substitution_failure_is_not_an_error}
\item \url{https://eli.thegreenplace.net/2014/sfinae-and-enable_if/}
\item \url{https://riptutorial.com/cplusplus/example/3777/enable-if}
\end{itemize}


\end{document}