\documentclass[11pt,a4paper]{article}
\usepackage[utf8]{inputenc}
\usepackage{amsmath}
\usepackage{amsfonts}
\usepackage{amssymb}
\usepackage{graphicx}
\usepackage{xcolor}
\usepackage{listings}
\usepackage{hyperref}


\lstset{
	language=C++
	,showlines=true
	,showstringspaces=true
	,tabsize=2
	,frame=single
	,breaklines=true
	,framexleftmargin=2mm
	,backgroundcolor=\color{white}
	,xleftmargin=2mm
	%, frameround={tttt}
	,captionpos=b
	,basicstyle=\tt \scriptsize
	,keywordstyle=\bfseries\ttfamily\color[rgb]{0,0,1}
	,numberstyle=\scriptsize,
	,identifierstyle=\ttfamily
	,commentstyle=\color[rgb]{0.133,0.545,0.133}
	,stringstyle=\ttfamily\color[rgb]{0.627,0.126,0.941}
	,showstringspaces=false
	,numbers=none
%	numbers=left,
	,stepnumber=1
	,numbersep=10pt
	,tabsize=4
	,breaklines=true
	,breakatwhitespace=false
	columns=flexible
%	upquote=true,
	extendedchars=true
	,xleftmargin=0pt
	,language=bash
   ,frame=single
}



\usepackage[left=2.00cm, right=1.50cm]{geometry}
\author{S. Kramm}
\title{Sfinae notes}
\date{\today}
\begin{document}
\maketitle

reference: \url{https://en.wikipedia.org/wiki/Substitution_failure_is_not_an_error}


\url{https://eli.thegreenplace.net/2014/sfinae-and-enable_if/}



Exemple 1: fonction qui renvoie un type dépendant du type en template

\begin{lstlisting}

template<typename T>
typename std::enable_if<
   ( std::is_same<T, int>::value ||
    std::is_same<T, char>::value ),
    T>::type
myFunc(T data)
{
	std::cout << "myFunc() data=" << data << "\n";
	return data+1;
}

int main()
{
    std::cout << myFunc( 42 );  // ok, affiche 43
    std::cout << myFunc( 'a' );  // ok, affiche 'b'
    std::cout << myFunc( 3.14 );  // Erreur !        
}

\end{lstlisting}



Exemple 2: fonction avec un argument

\begin{lstlisting}
#include <iostream>

struct A
{
    int a;
    A() { a=42; }
};

// for A& or const A&
template<
	typename T,
	typename std::enable_if<
		(std::is_same<T,A>::value || std::is_same<T,const A>::value),T
	>::type* = nullptr
> 
A&
func ( A& a1, T& a2 )
{
    std::cout << "template 1\n";
    return a1;
}

// for all other types

template<
	typename T,
	typename std::enable_if<
		(!std::is_same<T,A>::value & !std::is_same<T,const A>::value),T
	>::type* = nullptr
> 
A&
func ( A& a1, T a2 )
{
    std::cout << "template 2 a2=" << a2 << "\n";
    return a1;
}
 
int main()
{
    A a1,a2;
    A a3 = func(a1,a2);
    A a4 = func(a1,3);
}
\end{lstlisting}





\end{document}